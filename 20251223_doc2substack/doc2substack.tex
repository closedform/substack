\documentclass[11pt]{article}
\usepackage{amsmath,amssymb,amsthm}
\usepackage{geometry}
\usepackage{bm}
\usepackage{hyperref}
\usepackage{xcolor}

\geometry{margin=1in}

\title{\textbf{doc2substack: A Bridge for Mathematical Writing}}
\author{}
\date{}

\begin{document}
\maketitle

\begin{abstract}
If you write about physics/math/quantitative finance/etc., you know the pain of moving from your writing environment to Substack: you have a beautiful LaTeX document or Markdown file full of carefully crafted equations and when you try to paste it in, the math breaks... superscripts detach, logic symbols vanish, and complex equations turn into garbage.
\end{abstract}

\noindent \textbf{doc2substack} is a lightweight, open-source utility that converts your technical documents into clean, Substack-ready HTML. It preserves your prose, converts inline math to native Unicode for smooth reading, and renders complex display math as crisp, high-DPI images.

\section{Where to Get It}
The utility is open source and available directly in our repository:
\url{https://github.com/closedform/substack/tree/main/utils}

\section{How to Use It}

\begin{enumerate}
    \item \textbf{Write naturally:} Write your post in LaTeX (\texttt{.tex}), Markdown (\texttt{.md}), or any other Pandoc-supported format (\texttt{.docx}, \texttt{.ipynb}, etc.).
    \item \textbf{Run the converter:}
    \begin{verbatim}
    python substack/utils/doc2substack.py my_post.tex
    \end{verbatim}
    \item \textbf{Publish:} Open the generated \texttt{.html} file, copy everything, and paste it into the Substack editor.
\end{enumerate}

\section{Example: The Heat Equation}
The following section demonstrates \texttt{doc2substack} by rendering a standard PDE derivation. Notice how numbered equations are preserved and rendered as images.

Consider the one-dimensional heat equation for a function $u(x,t)$ representing temperature distribution over a rod:
\begin{equation}
    \frac{\partial u}{\partial t} = \alpha \frac{\partial^2 u}{\partial x^2}
\end{equation}
where $\alpha$ is the thermal diffusivity constant. The goal is to find a solution using the method of separation of variables. Let $u(x,t) = X(x)T(t)$. Substituting this form into the PDE yields:
\begin{equation}
    \frac{1}{\alpha T} \frac{dT}{dt} = \frac{1}{X} \frac{d^2X}{dx^2} = -\lambda
\end{equation}
Here, $\lambda$ is the separation constant. This splits the problem into two ordinary differential equations. The temporal component decays exponentially as $T(t) \propto e^{-\lambda\alpha t}$, while the spatial component satisfies:
\begin{equation} \label{eq:spatial}
    X''(x) + \lambda X(x) = 0
\end{equation}
For physical boundary conditions (e.g., Dirichlet conditions $u(0,t) = u(L,t) = 0$), the eigenvalues are discrete, given by $\lambda_n = (n\pi/L)^2$. The general solution is a superposition of these modes:
\begin{equation}
    u(x,t) = \sum_{n=1}^{\infty} B_n \sin\left(\frac{n\pi x}{L}\right) e^{-\alpha \left(\frac{n\pi}{L}\right)^2 t}
\end{equation}
The coefficients $B_n$ are determined by the initial condition $u(x,0)$ via Fourier sine series orthogonality.

\section*{Feedback}
If you find a bug or have a feature request, please \href{https://github.com/closedform/substack/issues}{open an issue on GitHub}. Or, just drop a comment below.

\vspace{1em}
\noindent Cheers,

\end{document}
