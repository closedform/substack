\documentclass[12pt]{article}
\usepackage{amsmath}
\usepackage{amssymb}
\usepackage{geometry}
\usepackage{hyperref}
\usepackage{graphicx}
\usepackage{newunicodechar}

% Map unicode characters to latex math for PDF compilation
\newunicodechar{α}{\ensuremath{\alpha}}
\newunicodechar{τ}{\ensuremath{\tau}}
\newunicodechar{≈}{\ensuremath{\approx}}
\newunicodechar{×}{\ensuremath{\times}}
\newunicodechar{ℏ}{\ensuremath{\hbar}}
\newunicodechar{ω}{\ensuremath{\omega}}
\newunicodechar{γ}{\ensuremath{\gamma}}
\newunicodechar{Δ}{\ensuremath{\Delta}}
\newunicodechar{·}{\ensuremath{\cdot}}
\newunicodechar{≤}{\ensuremath{\le}}
\newunicodechar{∝}{\ensuremath{\propto}}
\newunicodechar{μ}{\ensuremath{\mu}}
\newunicodechar{λ}{\ensuremath{\lambda}}
% Superscripts
\newunicodechar{⁰}{\textsuperscript{0}}
\newunicodechar{¹}{\textsuperscript{1}}
\newunicodechar{²}{\textsuperscript{2}}
\newunicodechar{³}{\textsuperscript{3}}
\newunicodechar{⁴}{\textsuperscript{4}}
\newunicodechar{⁵}{\textsuperscript{5}}
\newunicodechar{⁶}{\textsuperscript{6}}
\newunicodechar{⁷}{\textsuperscript{7}}
\newunicodechar{⁸}{\textsuperscript{8}}
\newunicodechar{⁹}{\textsuperscript{9}}
\newunicodechar{⁻}{\textsuperscript{-}}

\geometry{a4paper, margin=1in}

\title{Aetherial Network Storage: A Fermi Problem Analysis}
\author{Closed Form}
\date{\today}

\begin{document}

\maketitle



\section*{Aetherial Network Storage}
One of my favorite pastimes is swinging by the central team's office looking for a diversion (read: to distract everyone from actual work). On this particular visit, I found \href{https://www.linkedin.com/in/rafi-matchen-4331b3156/}{Rafi Matchen} deep in the weeds of lockless data structures. Something about the conversation, maybe the interplay of physics and computing or just the right amount of caffeine, got the creative juices flowing, and we started riffing on Fermi problems.

Here are a few that came out of that session. Feel free to steal them for your next interview.

\section{Introduction}
The internet is a physical object. It is a sprawling machine spanning the globe, built from glass, copper, and silicon. And like any physical machine, it has mass, momentum, and (surprisingly) memory, even when nothing is being written to a disk.

By treating the global network as a \textit{reservoir} of light and electricity, we can derive some surprising properties. We ask four simple questions:
\begin{enumerate}
    \item How much data is floating in the wires right now?
    \item Does this information weigh anything?
    \item Could you build a computer that stores nothing, but remembers everything?
    \item When will the internet become heavy enough to collapse into a black hole?
\end{enumerate}

\section{How Much Data Is Floating in the Wires Right Now?}

To estimate the amount of data stored in the global network infrastructure at any given instant ("data in flight"), imagine the internet as a continuous flow system. If we treat the global network as a pipe, the total volume depends on the flow rate and the length of the pipe.

High-level statistics often cite annual traffic in zettabytes, but this obscures the instantaneous state. We build a bottom-up estimate based on active usage.

\subsection{The Model}
We define the instantaneous storage S as the product of aggregate bandwidth and latency:
\[
S \approx R_{\text{total}} \times \tau
\]
where:
\begin{itemize}
    \item \textbf{Global Throughput (R\textsubscript{total}):} The sum of all active data streams.
    \item \textbf{Residency Time (τ):} The average time a bit spends in transit (or buffered) before vanishing.
\end{itemize}

\subsection{Estimating Throughput}
Let N be the global population, α be the fraction of active users (humans or automated agents) at any instant, and r be the average bandwidth per active user.
\[
R_{\text{total}} = N \cdot \alpha \cdot r
\]
If we assume a population N ≈ 8 billion, and an active fraction α ≈ 0.5 (accounting for the billions of phones, smart fridges, and servers that never sleep), and a blended average bandwidth of r ≈ 25 Mbps (balancing from someone streaming 4K video to a thermostat checking its network connection), we get:
\[
R_{\text{total}} \approx (4 \times 10^9) \times (25 \times 10^6 \text{ bit/s}) = 100 \text{ Pbps}
\]

\subsection{The Result}
How long does a bit live in the wires? Let's assume that on average, a piece of data travels halfway around the world before it hits its target. At the speed of light in fiber (about 2/3 c), plus a little time spent waiting at routers, we can estimate a life-span of τ ≈ 0.16 seconds.
Using this, the total in-flight data becomes:
\[
S \approx (12.5 \text{ PB/s}) \times (0.16 \text{ s}) = 2 \text{ PB}
\]
The "memory" of the internet's wires and buffers holds approximately \textbf{2 Petabytes}.

\section{Does Information Weigh Anything?}

So we have roughly S ≈ 2 PB of data floating around the world. Does it weigh anything? That depends entirely on how you define "mass." We can look at this through two distinct lenses: Information Theory and General Relativity.

\subsection{The Thermodynamic Limit (Landauer)}
In the purest abstract sense, information is entropy. Landauer's Principle dictates the minimum energy required to differentiate (or erase) a bit at temperature T:
\[
E_{bit} = k_B T \ln 2
\]

The total mass equivalent is derived from Einstein's E=mc\textsuperscript{2}:
\[
m_{info} = \frac{N_{bits} \cdot k_B T \ln 2}{c^2}
\]

For 2 PB of data (about 1.6 × 10\textsuperscript{16} bits) at room temperature (300 K):
\[
m_{info} \approx 5 \times 10^{-22} \text{ kg}
\]

This is negligible: roughly the mass of 300,000 protons.

\subsection{The Relativistic Limit (Signal Energy)}
But here's the catch: the internet isn't made of abstract bits. It's made of light. In optical fibers, a bit is represented by a pulse of roughly 10\textsuperscript{4} photons.

The energy of a single bit is:
\[
E_{signal} = n \cdot \hbar \omega
\]

The mass of the data is the mass-equivalent of the light pulses currently in transit:
\[
m_{signal} = \frac{N_{bits} \cdot E_{signal}}{c^2}
\]

With typical 1550 nm photons (about 10\textsuperscript{-19} J each), the total energy in flight is roughly 20 Joules:
\[
m_{signal} \approx 2.3 \times 10^{-16} \text{ kg}
\]

This is strikingly small, but macroscopic: it is roughly the mass of a single \textbf{E. coli bacterium}.

\section{A Computer That Stores Nothing, But Remembers Everything}

If the internet holds 2 PB of data, could it be used as a computer?
Imagine a machine that has no internal storage (RAM/HDD). Instead, it uses the network itself as a delay-line memory.

\subsection{The Passthrough Machine}
Consider a "head" that sits on a fiber optic loop of length L. It reads bits from the input, looks for simple patterns (like a word in a book), modifies them, and writes them to the output.
The "tape" is the loop itself. A bit written to the network travels distance L and returns to the input some time Δt later.

\subsection{The Kinetic Memory}
The effective memory capacity M of this machine is determined strictly by the geometry of the loop and the speed of signal propagation v:
\[
M = B \times \frac{L}{v}
\]
where B is the bandwidth of the head. Let's calculate the memory for a standard 1 Gbps connection (B = 10⁹ bits/s) using two distinct loops:

\textbf{Case A: The Earth Loop.} 
A fiber optic cable wrapped around the equator (L ≈ 40,000 km). The signal travels at v ≈ 200,000 km/s (speed of light in glass).
\[
\Delta t_{\text{earth}} = \frac{40,000 \text{ km}}{200,000 \text{ km/s}} = 0.2 \text{ s}
\]
\[
M_{\text{earth}} \approx (1 \text{ Gbps}) \times (0.2 \text{ s}) = 200 \text{ Mb} = 25 \text{ MB}
\]

\textbf{Case B: The Moon Loop.}
A laser link to the Moon and back (L ≈ 770,000 km). The signal travels at c ≈ 300,000 km/s (vacuum).
\[
\Delta t_{\text{moon}} = \frac{770,000 \text{ km}}{300,000 \text{ km/s}} \approx 2.56 \text{ s}
\]
\[
M_{\text{moon}} \approx (1 \text{ Gbps}) \times (2.56 \text{ s}) \approx 2.56 \text{ Gb} = 320 \text{ MB}
\]

\textbf{Case C: The iPhone Loop.}
To match the 8 GB RAM of a modern smartphone, our delay line needs to hold 64 × 10⁹ bits. At 1 Gbps, this requires a delay of Δt = 64 seconds.
\[
L_{\text{iphone}} = c \times 64 \text{ s} \approx 19.2 \text{ million km}
\]
Usefully, this is roughly 64 light-seconds, or a loop extending \textbf{1/8th of the distance to the Sun}.

\subsection{Conclusion}
A single "Passthrough Machine" using the Earth as a delay line accesses a \textbf{kinetic memory of 25 MB}, sufficient to run a lightweight operating system like Windows 95. To run a modern smartphone OS, your fiber optic loop would need to reach well past the Moon, extending a significant fraction of an Astronomical Unit.

\section{When Will the Internet Collapse Into a Black Hole?}

Finally, we ask: What is the \textit{minimal} loop size required for the internet to collapse into a black hole (Kugelblitz)?
We assume the mass of the wire itself is negligible (perhaps supported by antigravity), so gravity assumes only the mass of the data. However, the wire has \textit{volume}, which limits how tightly we can pack it.

\subsection{The Density Crossover}
For a black hole to form, we must pack the fiber into a ball of radius R that lies inside its own Schwarzschild radius (R ≤ R\textsubscript{s}).
But we cannot compress the fiber to zero size; we are limited by the physical volume of the glass. The most compact configuration is a solid sphere of cladding.

We have a "density race":
\begin{itemize}
    \item \textbf{Data Mass (M):} grows linearly with length L.
    \item \textbf{Event Horizon (R\textsubscript{s}):} grows linearly with Mass, so R\textsubscript{s} ∝ L.
    \item \textbf{Wire Ball Radius (R\textsubscript{ball}):} grows with the cube root of volume (L\textsuperscript{1/3}).
\end{itemize}

At short lengths (like 1 meter), the physical volume of the wire (R\textsubscript{ball}) is vastly larger than the tiny event horizon of the light (R\textsubscript{s}). But since R\textsubscript{s} grows faster (L) than the ball radius (L\textsuperscript{1/3}), there is a critical length where the horizon eventually swallows the wire.

\subsection{The Calculation}
Using a standard fiber diameter (d ≈ 125 μm) and our 1 Gbps linear mass density (λ ≈ 4.8 × 10⁻³⁶ kg/m):
\begin{enumerate}
    \item \textbf{Required Mass}: We need the Schwarzschild volume to equal the Wire volume:
    \[
    V_s \propto L^3 \quad \text{and} \quad V_w \propto L
    \]
    \item \textbf{The Result}: The crossover happens at approximately:
\end{enumerate}
\[
L_{\text{crit}} \approx 2 \times 10^{89} \text{ meters}
\]
This describes a spool of fiber 10⁶³ times the width of the Observable Universe.

Even if we could shrink the wire down to the \textbf{Planck length} (10⁻³⁵ m), the length required would still be 10⁵⁸ meters: roughly 10³² times the size of the universe.

Because the energy density of a 1 Gbps signal is so incredibly low, you need a "ball of yarn" unimaginably larger than the cosmos before the gravity of the light becomes strong enough to crush the medium carrying it.

\section{Summary}
\begin{enumerate}
    \item \textbf{Global Data in Flight:} ≈ 2 PB.
    \item \textbf{Mass of Information:} ≈ 2.3 × 10\textsuperscript{-16} kg (single bacterium).
    \item \textbf{Kinetic Memory:} 25 MB (Earth Loop) to 320 MB (Moon Loop).
    \item \textbf{Black Hole Limit:} A loop 10⁸⁹ meters long.
\end{enumerate}


\end{document}
